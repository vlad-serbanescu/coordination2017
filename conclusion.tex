\section{Conclusions}
In this paper we proposed a Java library for efficiently implementing the cooperative scheduling behavior of ABS. We observed significant improvements due to the trade-off of saving contexts as continuations in memory instead of native Java threads. Our target for this library is modeling real case studies such as the auctioning agent system presented or applications that already use the formal description capabilities of ABS such as the work of Hähnle and Muschevici in validating railway systems\cite{railway}. To provide support for the functional data types required by these case studies we plan to extend the backend to the Scala programming language together with this library. We want to study how well these models execute into the Java Runtime Environment compared to their direct implementations without the use of modeling languages. Furthermore having a portable JVM library library gives us a basis for industrial adoption of the ABS language.  Finally this provides an opportunity to use ABS rich features in formal verification,  resource analysis and deadlock detection as well as extensions that support real-time and distributed programming, in a software development context. 
%We have a java library
%We have already benchmarked overhead, but we want to use it for real case studies auctioning and incremental evaluation of railway systems
%Promising candidate fo a basis for industrial adoption of the ABS language.
%Advantage 
%ABS is a rich language with this backend because of the formal analysis, resource analysis and deadlock detection.
%Fromally defined language which provides the use of FA. 
%Extensions for real-time programming and distributed programming.  
%Scala backend for ABS have a full implementation of the functional data types (auctioning system)