
%%%%%%%%%%%%%%%%%%%%%%% file typeinst.tex %%%%%%%%%%%%%%%%%%%%%%%%%
%
% This is the LaTeX source for the instructions to authors using
% the LaTeX document class 'llncs.cls' for contributions to
% the Lecture Notes in Computer Sciences series.
% http://www.springer.com/lncs       Springer Heidelberg 2006/05/04
%
% It may be used as a template for your own input - copy it
% to a new file with a new name and use it as the basis
% for your article.
%
% NB: the document class 'llncs' has its own and detailed documentation, see
% ftp://ftp.springer.de/data/pubftp/pub/tex/latex/llncs/latex2e/llncsdoc.pdf
%
%%%%%%%%%%%%%%%%%%%%%%%%%%%%%%%%%%%%%%%%%%%%%%%%%%%%%%%%%%%%%%%%%%%


\documentclass[runningheads,a4paper]{llncs}

\usepackage{amssymb}
\setcounter{tocdepth}{3}
\usepackage{graphicx}
\usepackage{listings,calc}

\lstset{language=Java, numbers=left, columns=fullflexible, keepspaces=true, basicstyle=\scriptsize}

\usepackage{url}

\urldef{\mailsa}\path|{vlad.serbanescu, frank.s.de.boer, jaghouri}@cwi.nl|  
  
\newcommand{\keywords}[1]{\par\addvspace\baselineskip
\noindent\keywordname\enspace\ignorespaces#1}

\begin{document}

\mainmatter  % start of an individual contribution

% first the title is needed
\title{Cooperative Scheduling Behaviour from Modeling to Runtime}

% a short form should be given in case it is too long for the running head
\titlerunning{Cooperative Scheduling}

% the name(s) of the author(s) follow(s) next
%
% NB: Chinese authors should write their first names(s) in front of
% their surnames. This ensures that the names appear correctly in
% the running heads and the author index.
%
\author{Vlad Serbanescu
\and Frank de Boer \and Mahdi Jaghouri}
%
\authorrunning{Serbanescu et al.}
% (feature abused for this document to repeat the title also on left hand pages)

% the affiliations are given next; don't give your e-mail address
% unless you accept that it will be published
\institute{Leiden Institute of Advanced Computer Science\\
Centrum Wiskunde and Informatica\\
\mailsa\\
\url{http://www.cwi.nl}}

%
% NB: a more complex sample for affiliations and the mapping to the
% corresponding authors can be found in the file "llncs.dem"
% (search for the string "\mainmatter" where a contribution starts).
% "llncs.dem" accompanies the document class "llncs.cls".
%

\toctitle{Lecture Notes in Computer Science}
\tocauthor{Authors' Instructions}
\maketitle


\begin{abstract}
%Modeling applications at the design phase of any project is highly important in order to build a reliable, fast and robust system. Understanding the control flow of execution from just the model is crucial for the applications next software engineering phases such as implementation, testing, release and maintenance.  If we add that a large portion software systems are running several jobs in parallel, with a good model we can observe data consistency and detect deadlocks efficiently. One of the toughest models to execute, especially in a parallel or distributed environment is the Actor-based model which introduces the notion of cooperative scheduling, a high-level synchronization mechanism which allows an actor to continue to execute messages from its queue when the current message execution is suspended . In this paper we will investigate some of the challenges of translating the cooperative scheduling behavior into the Java Runtime Environment. We will analyze the Abstract Behavioral Specification (ABS) Language concepts which are very well suited for Agent-based applications modeling and provide a benchmark for testing several solutions of efficiently running cooperative scheduling behavior. 
Actor-based models of computation in general assume a run-to-completion mode
of execution of the messages.
The Abstract Behavioral Specification (ABS) Language extends the Actor-based model
with  a high-level synchronization mechanism which allows actors to suspend the
execution of the curren message and schedule in a cooperative manner another
queued message.  This  extension  is a powerful means for the expression and analysis
of fine-grained run-time dependencies between messages.

In this paper we introduce  and evaluate a new run-time system in Java for  the use of ABS as a full-fledged programming language.
The main challenge is the development of an efficient and scalable  thread-based implementation of cooperative scheduling.
By means of a typical benchmark we evaluate our proposed solution and compare it
to several other thread-based implementations of cooperative scheduling.







\keywords{Cooperative Scheduling, Mult-Agent Systems, High-Performace Computing, Language Design}
\end{abstract}


\section{Introduction}

-high-level description of cooperative scheduling\\
-position the paper with scala and Akka, state of the art\\
- problem statement: how stacks need to be saved in OO, how expensive a thread is\\
- strongly typed actors messages.
- the scope of the paper is implementation in the Java Runtime Environment/Java Virtual MAchine(JVM)
- using ABS as a mute programming language, real software development context, support for object-oriented design and foreign lanuage interface.


\section{ABS Language Concepts}
\label{lang}
Our reference modeling language for analyzing cooperative scheduling is ABS \cite{abs}, an actor-based modeling language whose semantics offers high-level synchronization mechanisms for parallel and  distributed applications. This is a very powerful modeling language with extensive support for concurrent programming \cite{cog}, resource analysis \cite{saco}, deadlock analysis \cite{dead} and remote communication \cite{dis,cloud}. Our focus in this paper are  high-level constructs for modeling asynchronous communication between actors using messages and cooperative scheduling of these messages by an actor.
%Mahdi:
{\bfseries the following are not the `focus of this paper' as suggested in the previous sentence}
The second construct can have optional annotations that define custom schedulers in order to satisfy an actor's specific behavior. Asynchronous method calls can also be annotated with costs and deadlines which can be combined with cooperative scheduling policies to create a very powerful scheduler.

\subsection{Asynchronous Method Invocation}\label{amc}
In ABS all actors communicate with each other using asynchronous method invocation.
This is written as ``a!m()" which sends a message to another actor ``a" to execute method ``m". 
The semantics additionally allows for synchronous method calls that only change the internal state of an actor $\leftarrow$ {\bfseries what does this mean?}. 
%ABS also has support for grouping Actors into Concurrent Object Groups (COG) that allow synchronous communication between actors of the same COG, but for the scope of our analysis and implementation simplicity we will assume that each actor is defined in its own group.  <-- then I would drop it from here
Each ABS actor has a queue that stores all invocations coming from other actors as messages that are to be executed. 
Each actor executes the messages sequentially in its own context while all actors run in parallel.  
The result of an asynchronous method invocation can be captured in an future which the caller actor can use to retrieve the result, but also to block its current execution in order to synchronize with the callee actor. 
This is done through the ``future.get" statement that a future supplies. % and has the same semantics as futures in other modeling and programming languages.  
One may also send the future as parameter to a message, allowing other actors to use the results.

An important observation to make here is that in modeling an application, ABS assumes there is no ``message overtaking", that is the order of messages delivered from one particular actor to another is the same as the order in which they were generated. 
This is a constraint that our implementation needs to take into account and a synchronization mechanism as will be described in our modeled application later in this section $\leftarrow$ {\bfseries do we? make sure it is not overemphasized here.}. 


\subsection{Await Construct}
The ``await" statement of ABS offers a high level mechanism that controls execution within an actor based on its internal state or the availability of the result of an asynchronous call via futures. A statement like "await future?" differs from "future.get" as it does not block the entire execution of the actor, but instead block the current message from which the ``await" statement was called and allows the actor to continue with executing other messages that are available in its queue. Similarly, for example "await this.boolVar" will suspend the current message executing this statement until the field boolVar is set to true (while another method is being executed). In both uses of the "await" construct the suspended message will be made available once the future is completed or the boolean condition evaluates to true. 
Even though a very simple concept, such a construct can simplify the logic of the program significantly. The effectiveness of this mechanism will be illustrated within the example in Section \ref{ag}. 
Furthermore, we will see some of the challenges encountered when implementing this behavior in Sections \ref{comp} and \ref{run}.        

- the main concepts: async calls, await on boolean and futures, each object has its own queue.
- example for coroutines (Paul Klint paper at SEN)

\subsection{Proof of concept}
\label{ag}
Here we illustrate the use of ABS in implementing Agent-based modeling languages.
Agent-based modeling has been shown to be a powerful means to express organizational abstractions of autonomous behavior, including auctioning systems \cite{agent_auction,bas16}.
However, it is still a major challenge to generate  \emph{production} code from  high-level agent-based modeling concepts, e.g., the deliberation cycle which integrates  goal-oriented computation with an event-based communication approach. 
To lower the barrier for adoption by industry, Dastani and Testerink \cite{bas16} propose a Java methodology which guides the development of agent-based models. 
This includes a corresponding library of object-oriented design patterns for modeling agent-based concepts, called OO2APL. 

OO2APL  includes a complex middleware for management of the deliberation cycle and the event-based communication between agents. 
Furthermore, this middleware is tightly coupled with the high-level design patterns for the deliberation cycle and  the event-based communication mechanism.
In contrast,  Listing \ref{list:Agent} shows that modeling agents directly by actors  allows to abstract in a concise manner from the underlying implementation of the deliberation cycle  and the event-based communication mechanism.
It provides  a high-level design pattern in ABS capturing the code structure for modeling agents with a clear separation of message plans and goal rules. 
The first three lines introduce the generic data types for goals, beliefs and messages.
In ABS, one can define the goals, beliefs and messages as abstract data types.
This makes it easy to define the plans for  handling incoming messages and the rules for processing the goals by means of pattern matching against these model-specific data types (lines 14 and 21, respectively, of Listing \ref{list:Agent} ).
The general interface of an agent simply consists of a single method for receiving  messages. An  implementation of this interface consists of a set of beliefs and a set of goals which are initialized upon creation by means of a statement
\lstinline|Agent a1= new A1(B,G)|.
This statement  assigns to the variable a1 a dynamically generated  identity of the newly created agent and initializes its beliefs and goals (by  the actual parameters \lstinline|B| and \lstinline|G|).
Finally,  the \lstinline|run()| method (line 27 of  Listing \ref{list:Agent}) ensures proactive processing of the goal rules and allows for
interleaved processing of the messages by means of cooperative scheduling as invoked by the \lstinline|suspend| statement.
Thanks to the flexibility of ABS, one can easily adopt variations of the above design pattern.
For example, placing the \lstinline|suspend| statement inside the \lstinline|for| loop will allow interleaving individual goal rules with message plans.
Furthermore, application-specific scheduling policies \cite{rabs,cog}  may be applied if desired, for example, to give some goals or messages higher priorities.



\begin{lstlisting}[caption= Generic Agent Model, label=list:Agent]
data Goal = Goal;
data Belief = Belief;
data Message = Message;

interface Agent {
	Unit message(Message m);
}

class A1(Set<Belief> init_beliefs, Set<Goal> init_goals)  implements Agent {
	Set<Belief> beliefs = init_beliefs;
	Set<Goal> goals = init_goals;
	
	Unit message(Message m) {
		case m { 
			Message =>  { }
			_ => { }
		}
	}
	
	Unit goal_rule(Goal g) {
		case g {
			Goal => { }
			_ => { }
		}
	}
	
	Unit run() {
		 while(this.goals != EmptySet) {
  		    for (g in this.goals) {
		        this.goal_rule(g);
		    }
		  suspend;
		}
	}

}

\end{lstlisting}

%// example of a Main configuration
%{
%    Agent a1 = new A1(set[], set[Goal]);
%}





\begin{table}
\begin{tabular}{|l|c|c|}
	\hline
	& AuctioneerAgent & BiddingAgent \\
	\hline
Goals & Item1, Item2 & Item1 \\
Beliefs & costs, timing constraints & budget, timing constraints, risk factor \\	
Received messages (plans) & Bid(item, value) & Announce (item) \\
 & & Sold (item) \\
Goal Rules & start auction for the goal & \\
\hline
\end{tabular}
\end{table}



Listing \ref{list:agent} shows a concrete model of an agent  in an auctioning system.
Here, we define four types of messages sent and received by auctioning and bidding agents.  TODO: more explanation required, i.e., what is the initial goal?


\begin{lstlisting}[caption= Agent Model, label=list:agent]
data Message = 
Announce(Agent announceCaller, Item  toSell, Price price) | 
Bid(Agent bidCaller, Item  toBuy, Price bid) |
Sold(Agent soldCaller, Item soldItem) |
Result(Set<Agent> winners, List<Price> prices, List<Agent> unhappy);

class BiddingAgent(Set<Belief>  [] , Set<Goal > set[init_goal], Rat risk) implements Agent {
	Set<Belief>  beliefs = [];
	Set<Goal> goals = set[init_goal];
        Rat risk = risk;
	
	Unit message(Message m) {
		case m {
			Announce(caller, slot, price) =>  { ...   } 
		        Sold(caller, slot) => { ... }
		}
	}
}
\end{lstlisting}



\section{Cooperative Scheduling Implementation Schemes}
\label{scheme}
-sequence diagram of what happens in an actor during synchronous and asynchronous method calls.

\subsection{Modeling Language Concepts in the JVM}
In this section we will investigate the evolution of the scheme used to implement cooperative scheduling in the Java Runtime environment from a very simple approach to using several libraries and features that the latest version of Java provides. Conceptually an actor has one thread of execution, which means it can run only one method at a time. Practically, however, allocating a real thread for each actor is highly expensive, as we can have a very large number of actors in the modeled application. Very roughly speaking, an executor service in Java provides a queue of tasks and an efficient way of running those tasks on a few threads. Due to the optimizations provided by Java implementations, an executor in principle is the best way to scale to many concurrent tasks. We use the terms executor and thread pool interchangeably, referring to the main interface \cite{execserv} that facilitate parallel programming in Java. 

\paragraph{An Actor has a lock and Every Asynchronous call is a Thread}
The trivial straightforward approach for implementing cooperative scheduling in Java is to model each actor as an object with a lock for which execution threads compete. Each asynchronous call would then generate a new thread with it own stack and context. The disadvantage here is that we will need a lock per actor that must be checked by every message handler upon start, and freed upon completion. Whenever a control switch statement occurs the thread would be suspended by the JVM's normal behavior. When the release condition is enabled a suspended thread would become available and in turn compete with the other available threads in order to execute on the actor. The main performance drawback of this approach is the large number of threads that are created, which restricts any application from having more method calls than the main memory can support live threads.

\paragraph{Every Actor is a Thread Pool}
To reduce the number of live threads in an application such that it can run independent of the number of asynchronous calls, we can use Java 8 new features and model each invocation as a task using lambda expressions and organize each actor as a Thread Pool. This gives the actor an implicit queue to which tasks are submitted. We obtain a small reduction in the number of threads corresponding to the number of tasks that have been submitted, but not started.  Once they are started the threads still have to compete for the actor's lock in order to execute, but the number of live threads can be restricted to the number of threads allowed by each Thread Pool, while the rest of the invocations remain in the pool thread pool queue as tasks. 

\paragraph{Every System has a Thread Pool}
In the previous two approaches we modeled the concept of an actor being restricted to one task at a time by introducing a lock on which threads compete. Up to this point we still haven't discussed the issue of suspension and release points. However with all invocations modeled as tasks that don't need a thread before they start, we can simply assign one thread of execution to each actor and submit just task to this thread. After all, there is no point in starting more than one task only to have it stuck on the actor's lock. Therefore we eliminate the lock and thread pool-per actor concepts and introduce one thread pool per system. The task corresponding to an actor is then responsible for taking messages one by one from the queue and running them. This removes the requirement to synchronize every message handler, but it comes at the cost of having to manage message queues manually.
Now the threads are only limited to the number of actors in the system...or are they? When cooperative scheduling occurs, the executing thread will be suspended and therefore still live in the system so the application's live threads will be equal to the number of "await" statements in the program. The application will then be limited by the maximum number of suspended threads that can exist in the main memory.  Furthermore we still require a lock to ensure than upon release this thread will only execute when the actor to which it belongs is idle.

\paragraph{Synchronous calls context}
To eliminate the problem of having live threads when cooperative scheduling occurs we can simply use lambda expressions to turn the continuation into a method call and pass its current state as parameters to the method. Essentially what we do is allocate memory for the continuation on the heap, while it is suspended instead of holding a stack for it which is much more limited.  We can maintain this call into a separate queue of tasks that are "awaiting" either  on a condition or on a future and insert it in the queue of the actor once it is released. The problem with this approach is that we cannot pass a state that represents a chain of synchronous calls within the actor that eventually encounters an await. In this particular scenario we would still have to save the call stack as a thread and encounter (although to a smaller extent) the same problem as in the previous paragraph. At most, we can give an increased priority to the suspended threads that save call stacks to execute on actors once they are released.

\paragraph{Fully Asynchronous Environment}
The only limitation Java has now is how avoid thread explosion when a chain of method calls is suspended within an actor, or in simpler terms how to save this call stack without using a thread? To do this we can try to alter the bytecode to resume execution at runtime from a particular point, but we want our approach to be independent of the runtime and be extensible to other programming languages. Therefore we try to approach the problem differently, if we can turn a continuation that does not originate from a stack of calls into a task, is it possible to extend this to synchronous calls as well? We know that this issue arises when methods that contain an "await" statement are called synchronously, but at compile time we can easily identify all of these methods form the ABS code. We can simply mark these methods at compile time and transform them into asynchronous calls followed by an "await" statement on the future generated from the call. Now we can use the same approach for translating these continuations into tasks using lambda expressions and thus eliminating any suspended threads in the system. There remains only one important issue of how these particular continuations are scheduled, but this approach together with the process of translating continuations into tasks with be detailed in Section\ref{comp}. 


\subsection{Optimizations for the JVM}
As we have seen and will also prove using a benchmark in Section \ref{bench} we improve performance by eliminating the number of expensive threads in Java that are required in order to implement the cooperative scheduling model. Using this data-oriented approach for saving contexts we limit any application to the system's main memory size, but added to that we also obtain some other optimizations related to Java's features.

\paragraph{Demand-driven Approach}
An important advantage of having a task assigned to each actor and a manually processed queue is that we can start and stop the task depending on the queue state. This is simply done by any other actor who sends an invocation to an empty queue and subsequently the task stops when there are no more tasks in the queue. An important observation to make here is the two scenarios when a release point may occur. As ABS semantics do not allow actors to modify each other's internal state, we know that a release point that will validate a boolean condition based on a change of an actor's state can only occur during another task that is already executing on that actor. This release point will always occur before the running task ends and therefore the queue can never be empty as the released message will be available in the queue and no special notification will be required to resume the task assigned to the actor. On the other hand, release points may also occur when actors fulfill a future and therefore we require the system to have a notification mechanism for actors with empty queues and newly enabled messages which we will detail in Section \ref{run}. This requires maintaining a global hash table, mapping every future to the set of actors that are awaiting on its completion.

\paragraph{Optimal Usage of System Threads}
The approach of using one thread pool per system gives the user direct control over the number of live threads the application creates. Using the Executor interface in Java allows the user to choose the type of thread pool that manages the actors and set the maximum number of threads that are available. For example the user can limit the number of threads the application has to the number of cores that the machine provides and avoid context switches made by the JVM. This is turn means that the implementation has to provide fair usage of the threads to the tasks that run the actors, an issue which we will also touch upon in Section \ref{run}.  

\paragraph{Eliminating Busy Waiting}
Cooperative scheduling through the "await" statements may suspend the current message run by the actor based on either a particular inner stare or future requiring completion on a different actor. We discussed how the task assigned to the actor can start or stop based on release points, but how exactly does an actor verify that a release point has completed ? Clearly having a task continuously iterate through all suspended messages (busy-waiting) is inefficient and while we can permanently mark a message that needs a future to complete as available, we cannot do that for a message which is released by particular valid state, as it the state can change by the time it is run. Instead we assign this verification to the task that fetches messages from the queue, and simply stop the task if no message is available. If the task does stop, it means that the actor is in a state in which it is unable to execute any of its messages and requires another actor to either send a new invocation that will change its state or the system to send a notification about a future that may release some of its messages and re-enable the task assigned to it.  


\paragraph{Using JVM Garbage Collection}
Using the approach explained so far in this section, the only extra references we need for the actors are the ones inside the global hash-table required for the notification mechanism. Once the future is completed and notifications are sent the key is deleted and the actor references become unreachable. Therefore we can leave the entire garbage collection process to the Java Runtime Environment as no other bookkeeping mechanisms are required.


\section{Continuation Generation at Compile Time}

- Mahdi's blog post 
- formal explanation for creating continuations.
\section{Runtime Behaviour}
\label{run}
The implementation of our Cooperative Scheduling is done through a Java library which contains a set of classes and interfaces that have a direct mapping to the ABS language concepts described in Section \ref{lang}. The library provides an implementation of the actor's cooperative scheduling behavior, the suspension and release mechanism and the fully asynchronous environment while respecting the logic of synchronous calls. The library provides the solution and the optimizations discussed in Section\ref{scheme} using the preprocessed continuations generated as detailed in Section \ref{comp}. The library can be obtained as a maven project and is available online\cite{library}

\subsection{Deployment Component}
ABS uses the concept of Deployment Component to decri a system or machine on which actor's run. Therefore in our library we use this class that manges the two elements that the solution requires at the level of the system. The first is the system executor which currently in the the library is a \textit{newFixedThreadPool} singleton ExecutorService initialized with a fixed number of threads equal to the number of cores in the system. An actor may start a new Main Task by simply calling the static method \lstinline|DeploymentComponent.submit(new MainTask())| offered by the Deployment Component. Inside the class there is support to safely call \lstinline|DeploymentComponent.shutdown()| the ExecutorService when all the actors in the application have completed execution, depending of course if the logic of the program reaches a point when all actors have finished execution. The second element is the notification mechanism together with a \textit{ConcurrentHashMap} that contains mappings of futures that hold release points on actors in the system. Actors that complete a certain future can call the static method \lstinline|DeploymentComponent.releaseAll(f)| to notify actors that contain messages suspended on that particular future. 

%A problem that can arise when there are more actors than available threads in the system's executor, some actors may keep processing one message after the other from its queue, and thus keeping the thread it is running on, while some other actors are starving, i.e., not being assigned to any thread.

%\subsection{Suspension and Release Points}
%To implement cooperative scheduling our library provides abstractions for guards that control suspension and release points. The are supported through the abstract class \textbf{Guard} and its subclasses \textbf{FutureGuard}, \textbf{PureExpressionGuard} and \textbf{ConjunctionGuard} that describe the conditions on which an actor's message can await: either a future, a particular valid expression or a group of these conditions respectively. 

\subsection{Actor Implementation}
The library offers an interface \textbf{Actor} containing several methods for implementing the behavior of synchronous (\textit{sendSync}), asynchronous method (\textit{send}) calls and await statements (\textit{await}) from ABS. The library currently supports an implementation of this interface called \textbf{LocalActor} for actors running on the same machine. Inside this class the is a \textit{messageQueue} defined which holds all the invocations (synchronous or asynchronous) that have been submitted to the actor as tasks. The implementation defines an inner class \textit{MainTask} which is a Java Runnable that corresponds to the task responsible for taking messages one by one from the queue and running them. There is a problem when the queue is empty. Since we do not want to make this task busy-wait until a message arrives, an idea would be to check upon insertion of a new message into the queue whether such a task exists already. This, however, requires some careful synchronization. For every actor, we keep a local atomic boolean flag \textit{mainTaskIsRunning}. A first approach looks like this:

\begin{lstlisting}[caption= Basic Synchronization for the Demand-Driven Approach]
	// inside the task
	class MainTask implements Runnable{
		public void run() ({
				// iterate through queue and take one message and run it
			mainTaskIsRunning.set(false);
		}
	}
	// when inserting a new message
	messageQueue.add(m);
	if (!mainTaskIsRunning.compareAndSet(false, true)) {
		DeploymentComponent.submit(new MainTask())
	}...
\end{lstlisting}

The problem with the above code is that the check of the queue for emptiness and setting the flag to "false" is not atomic, and in between these two statements, a new message may be inserted into the queue without spawning a new Main Task. To remedy this, we need to introduce a new method that can check the queue for emptiness and set the flag to "false" in a critical section, for example inside a "synchronized" block using the \textit{messageQueue} or \textit{mainTaskIsRunning} as the lock. Additionally, either the insertion into the \textit{messageQueue} or \textit{compareAndSet} of \textit{mainTaskIsRunning} should also use the same lock obviously.

\begin{lstlisting}[caption= Complete Synchronization for the Demand-Driven Approach]

private boolean takeOrDie() {
	synchronized (mainTaskIsRunning) {
		// iterate through queue and take one ready message 
		// if it exists set the next message for the main task and then
		return true;
		//if the queue if empty or no message is able to run
		mainTaskIsRunning.set(false);
		return false;
	}
}

private boolean notRunningThenStart() {
	synchronized (mainTaskIsRunning) {
		return mainTaskIsRunning.compareAndSet(false, true);
	}
}

// inside the task
class MainTask implements Runnable{
	public void run() {
		while (takeOrDie())
			// proceed to take the next message message and run it	 
	}
}

// when inserting a new message
messageQueue.add(m);
if (notRunningThenStart()) {
	DeploymentComponent.submit(new MainTask());
}
\end{lstlisting}

Another problem with the approach above is that (when there are more actors than available threads), some actors may keep processing one message after the other from its queue, and thus keeping the thread it is running on, while some other actors are starving, i.e., not being assigned to any thread. To remedy this, we could change the while loop to an if statement like this:

\begin{lstlisting}[caption= Fairness Between Actors]

// inside the task
class MainTask implements Runnable{
	public void run() {
		if (!takeOrDie())
			return;
		// proceed to take the next message message and run it	 
		DeploymentComponent.submit(this);
	}
}
\end{lstlisting}






\section{Benchmarking the Implementation Schemes}
\label{bench}
The main problem that we encountered  when implementing cooperative scheduling was saving the context of an execution and resuming from that context. To do this in Java using threads and context switches is simply too expensive and heavily limits the application to the number of native threads that can be created. 

To measure the improvement provided by Java 8 features, we benchmark a
simple example that is illustrated in Fig.~\ref{bench:sf}. In the
example we have one actor containing an  which receives a large
number of messages stored in its queue. This message recursively calls
a function that creates a stack frame after which a message is
sent to a different Actor to run in parallel a function that computes a large number of trigonometric operations . The object is then suspended to await the
result of this function, resulting in the requirement to save the stack frame in order to allow the next message from the queue to run
on the actor.  We varied the total number
of messages in the object's queue to compare performance between a trivial thread based approach and our optimized solution in the Java backend for ABS. 

\begin{figure}
	\label{bench:sf}
	\centering
	\includegraphics[scale=0.6]{scenario}
	\caption{Cooperative Scheduling Benchmark Scenario}
\end{figure}

\par The results are shown in
Fig.~\ref{bench:jj}. The performance figures presented are for one
actor that is running 16--8192 method invocations, each with a
recursive stack frame of 5 and awaits the
completion of 10000 trigonometric functions before completing. 



\begin{figure}
	\label{bench:jj}
	\centering
	\includegraphics[scale=0.8]{jaj8}
	\caption{Perfomance figures of the two Java implementations for Cooperative Scheduling}
\end{figure}

\par These results do show that our solution mitigates the limitation of heavy native Java threads. However, while our solution is catered towards a widely-used language, it doesn't mean that there aren't other languages that are more suited to implement ABS language concepts efficiently and without these limitations. In Section \ref{coop} we listed several optimizations that were inferred from out implementation solution. What we want to do is compare this solution to an ABS backend implemented in Erlang that uses the same Thread-based approach but does not suffer from any limitation of native threads. We want to observe if our data-oriented approach can be comparable to Erlang's lightweight threads. The result in Figure \ref{bench:ej} show that our approach fares much better once the number of messages passes 1024. This result also strenghtens ABS purpose to provide a programming language for real applications.

\begin{figure}
	\label{bench:ej}
	\centering
	\includegraphics[scale=0.8]{erlj8}
	\caption{Perfomance figures of the Erlang and Java backends for Cooperative Scheduling}
\end{figure}




\section{Conclusions}
We have a java library
We have already benchmarked overhead, but we want to use it for real case studies auctioning and incremental evaluation of railway systems
Promising candidate fo a basis for industrial adoption of the ABS language.
Advantage 
ABS is a rich language with this backend because of the formal analysis, resource analysis and deadlock detection.
Fromally defined language which provides the use of FA. 
Extensions for real-time programming and distributed programming.  
Scala backend for ABS have a full implementation of the functional data types (auctioning system)





\begin{thebibliography}{4}

\bibitem{rabs}Joakim Bjork, Frank S. de Boer, Einar Broch Johnsen, Rudolf Schlatte, Silvia Lizeth Tapia Tarifa:
User-defined schedulers for real-time concurrent objects. ISSE 9(1): 29-43 (2013)

\bibitem{cgf}De Boer, Frank S., Dave Clarke, and Einar Broch Johnsen. "A complete guide to the future." European Symposium on Programming. Springer Berlin Heidelberg, 2007.

\bibitem{paj8} Nobakht, Behrooz, and Frank S. de Boer. "Programming with actors in Java 8." International Symposium On Leveraging Applications of Formal Methods, Verification and Validation. Springer Berlin Heidelberg, 2014.

\bibitem{bas16} Mehdi Dastani, Bas Testerink:
Design patterns for multi-agent programming. IJAOSE 5(2/3): 167-202 (2016)

\bibitem{abs} Johnsen, E. B., Hähnle, R., Schäfer, J., Schlatte, R., and Steffen, M. (2010, November). ABS: A core language for abstract behavioral specification. In International Symposium on Formal Methods for Components and Objects (pp. 142-164). Springer Berlin Heidelberg.

\bibitem{cog} Nobakht, B., de Boer, F. S., Jaghoori, M. M., and Schlatte, R. (2012, March). Programming and deployment of active objects with application-level scheduling. In Proceedings of the 27th Annual ACM Symposium on Applied Computing (pp. 1883-1888). ACM.

\bibitem{saco} Albert, E., Arenas, P., Flores-Montoya, A., Genaim, S., Gómez-Zamalloa, M., Martin-Martin, E., and Román-Díez, G. (2014, April). SACO: static analyzer for concurrent objects. In International Conference on Tools and Algorithms for the Construction and Analysis of Systems (pp. 562-567). Springer Berlin Heidelberg.

\bibitem{dead}Flores-Montoya, A. E., Albert, E., and Genaim, S. (2013). May-happen-in-parallel based deadlock analysis for concurrent objects. In Formal Techniques for Distributed Systems (pp. 273-288). Springer Berlin Heidelberg.

\bibitem{dis}Şerbănescu, V., Azadbakht, K., and de Boer, F. (2016). A java-based distributed approach for generating large-scale social network graphs. In Resource Management for Big Data Platforms (pp. 401-417). Springer International Publishing.

\bibitem{cloud} Bezirgiannis, N., and de Boer, F. (2016, January). ABS: a high-level modeling language for cloud-aware programming. In International Conference on Current Trends in Theory and Practice of Informatics (pp. 433-444). Springer Berlin Heidelberg.

\bibitem{agent_auction}Franco Zambonelli, Nicholas R. Jennings, Michael Wooldridge. 
Organisational Abstractions for the Analysis and Design of Multi-agent Systems
Agent-Oriented Software Engineering. Volume 1957 of the series Lecture Notes in Computer Science pp 235-251

\bibitem{execserv} https://docs.oracle.com/javase/8/docs/api/java/util/concurrent/ExecutorService.html

\bibitem{library}https://github.com/vlad-serbanescu/abs-api-cwi.git
\end{thebibliography}



\end{document}
