\section{Introduction}

The Actor-based model of computation \cite{Agha} is particularly tailored to the description of distributed systems.  Actors represent processes that execute in parallel and interact via the asynchronous communication of messages.
Messages in general are queued and processed in a run-to-completion mode of execution.
The  Abstract Behavioral Specification (ABS) \cite{abs}  Language  integrates the Actor-based model of distributed and parallel computation with object-orientation:
in ABS a message specifies a  call to one of the methods provided by the callee,
as such it supports the ``programming to interfaces'' paradigm (as, for example, in the Axtor-based modeling language Rebeca \cite{Sirjani}).
ABS further  extends the Actor-based model with  a high-level synchronization mechanism which allows actors to suspend the
execution of the current message and schedule in a cooperative manner another
queued message.  This  extension  is a powerful means for the expression and analysis
of fine-grained run-time dependencies between messages.

In this paper we introduce  and evaluate a new run-time system in Java for  the use of ABS as a full-fledged programming language which supports a Foreign Language Interface (FLI) for the import of Java librarries. The main challenge is the development of an efficient and scalable  thread-based implementation of cooperative scheduling
which supports a translation of ABS models into Java source code.
The basic feature of our proposed solution is the implementation of messages by means of
lambda expressions (as provided by Java 8), i.e., the method call specified in a message
is translated into a corresponding lambda expression which is passed and stored as
an object of type Callable or Runnable (depending on whether it returns  a result).
This  feature forms  the basis of a run-time system which manages
the sending, storage and execution of  the messages and their continuations
which arise because of the cooperative scheduling.
By means of a typical benchmark we evaluate our proposed solution and compare it
to several other thread -based implementations in Java of cooperative scheduling.

\paragraph{Related Work}

Our main motivation is to provide an implementation of the ABS language  as a full-fledged Actor-based programming language which supports,
as mentioned above,  the ``programming to interfaces'' paradigm.

There exist various implementation schemes for cooperative scheduling in ABS.
In the process-oriented approach sending a message  is implemented by the generation of a  corresponding process. This basic approach is followed in the Java backend of ABS \cite{abs,Schafer} and in the Erlang backend \cite{Erlang}.
In contrast, in the Haskell backend \cite{Haskell}  the use of continuations allows
to queue  a message and dequeue it for execution.





